\documentclass{article}
\usepackage[a4paper]{geometry}
\usepackage{graphicx} % Required for inserting images
\usepackage[ruled,linesnumbered]{algorithm2e}
\usepackage{amssymb}
\usepackage{mathtools}
\usepackage{tikz}
\usepackage{microtype}

\usepackage{enumitem}
\setlist[description]{leftmargin=2em,labelindent=2em}

\usepackage{polyglossia}
\setmainlanguage{english}

\usepackage{fontspec}
\setmainfont{EB Garamond}

\usepackage{amsthm}
	\theoremstyle{definition}
	\newtheorem{mytheorem}{Theorem}[section]
	\newtheorem{mylemma}[mytheorem]{Lemma}
	\newtheorem{myproposition}[mytheorem]{Proposition}
	\newtheorem{mycorollary}[mytheorem]{Corollary}
	\newtheorem{myconjecture}[mytheorem]{Conjecture}
	\newtheorem{mydefinition}[mytheorem]{Definition}
	\newtheorem{myassumption}[mytheorem]{Assumption}
	\newtheorem{mynotation}[mytheorem]{Notation}
	\newtheorem{myremark}[mytheorem]{Remark}
	\newtheorem{myexample}[mytheorem]{Example}
	\newenvironment{myproof}{\begin{proof}}{\end{proof}}

\usepackage[colorinlistoftodos,prependcaption,textsize=tiny]{todonotes}
\newcommand{\dkcom}[2][1=]{\todo[linecolor=purple,backgroundcolor=purple!25,bordercolor=purple,#1]{#2}}
\newcommand{\dknote}[2][1=]{\todo[linecolor=Plum,backgroundcolor=Plum!25,bordercolor=Plum,#1]{#2}}

\newcommand{\compl}{\mathrel{\backslash}}
\newcommand \ie {\textit{i.e.}}
\newcommand \id {\mathrm{id}}
\newcommand \tuple [1] {\left\langle #1 \right\rangle}
\newcommand{\funct}[5] {
	\begin{equation}
		\begin{aligned}
			#1:#2 & \longrightarrow #3 \\
			#4 & \longmapsto #5 
		\end{aligned}
	\end{equation}
}

\newcommand \Hyp {\mathrm{Hyp}}

\NewDocumentCommand \catspan   {mmmO{}O{}} {{#1 \xlongleftarrow{#4} #2 \xlongrightarrow{#5} #3}}
\NewDocumentCommand \catcospan {mmmO{}O{}} {{#1 \xlongrightarrow{#4} #2 \xlongleftarrow{#5} #3}}

\makeatletter
\providecommand{\leftsquigarrow}{%
  \mathrel{\mathpalette\reflect@squig\relax}%
}
\newcommand{\reflect@squig}[2]{%
  \reflectbox{$\m@th#1\rightsquigarrow$}%
}

%% tikz
\usepackage{quiver}
\usepackage{tikz}
\usetikzlibrary {cd,arrows.meta}
\usepackage{tikz-cd}
\usepackage{adjustbox}
\usetikzlibrary{positioning}
\usetikzlibrary{graphs}
\usetikzlibrary{backgrounds}
\usetikzlibrary{arrows.meta}
\newcommand \separator {\bigskip\hrule\bigskip}
\newcommand \hide [1] {\bgroup}
\tikzset{
	mgvertex/.style={
		circle,
		fill,
		inner sep=0pt,
		minimum size=5pt,
		label=#1:{\tikzgraphnodetext},
		execute at begin node=\hide,
		graphs/as={}
	},
	mgvertex/.default={above},
	mg/.style={
		graphs/every graph/.style = {grow right sep = 15mm},
		every edge quotes/.style = {draw,fill=white,sloped,midway,->},
		baseline = (current bounding box.center)
	},
	k1/.pic = {
		\graph {
			0 [mgvertex=left] -!- 1 [mgvertex=right] ;
			2 [mgvertex=left] -!- 3 [mgvertex=right] ;
		} ;
	},
	l1/.pic = {
		\graph {
			0 [mgvertex=left] -> ["a"] 1 [mgvertex=right] ;
			2 [mgvertex=left] -> ["b"] 3 [mgvertex=right] ;
		} ;
	},
	l2/.pic = {
		\graph {
			4 [mgvertex=left] -> ["a"] 5 [mgvertex=right];
		} ;
	},
	k2/.pic = {
		\graph {
			4 [mgvertex=left]-!- 5 [mgvertex=right];
		} ;
	}
}


\title{Group 2 - Critical Pairs for String Diagram Rewriting}
\author{Anna Matsui, Innocent Obi, Guillaume Sabbagh, Leo Torres,\\ Diana Kessler, Juan Meleiro, Koko Muroya}
\date{June 2024}

\begin{document}

\maketitle

\section{Notation} 
\textbf{Composition}. We use postfix notation for composition, separated by the symbol "$;$", so if $f$ and $g$ are morphisms in a category, then $f;g$ denotes the morphism obtained by first applying $f$ and then $g$. \\

\textbf{Coequalizer}. When we need to compute a coequalizer, we will write $(Q,q) = \mathtt{coeq}_A(f,g)$ to denote that $(Q, q: B \to Q)$ if the following diagram is a coequalizer:

% https://q.uiver.app/#q=WzAsNCxbMSwwLCJCIl0sWzAsMCwiQSJdLFsxLDFdLFsyLDAsIkMiXSxbMSwwLCJmIiwwLHsib2Zmc2V0IjotMn1dLFsxLDAsImciLDIseyJvZmZzZXQiOjJ9XSxbMCwzLCJxIl1d
\[\begin{tikzcd}
	A & B & C \\
	& {}
	\arrow["f", shift left=2, from=1-1, to=1-2]
	\arrow["g"', shift right=2, from=1-1, to=1-2]
	\arrow["q", from=1-2, to=1-3]
\end{tikzcd}\]

\textbf{Pullbacks}. Throught this paper, we will use the product-style notation for pullbacks. Given the following diagram in a category $\mathcal{C}:$ 

% https://q.uiver.app/#q=WzAsMyxbMSwxLCJDIl0sWzEsMCwiQiJdLFswLDEsIkEiXSxbMSwwLCJnIl0sWzIsMCwiZiIsMl1d
\[\begin{tikzcd}
	& B \\
	A & C
	\arrow["g", from=1-2, to=2-2]
	\arrow["f"', from=2-1, to=2-2]
\end{tikzcd}\]
 
we denote the limit of the above diagram by $A \times_C B$.

We also write $(C,f,g) = p_1\compl p_2$ to denote the pullback complement, i.e the triple $(C,f,g)$ that makes $P$ in the following diagram be a pullback.

% https://q.uiver.app/#q=WzAsNCxbMSwxLCJDIl0sWzEsMCwiQiJdLFswLDEsIkEiXSxbMCwwLCJQIl0sWzEsMCwiZyJdLFsyLDAsImYiLDJdLFszLDEsInBfMiJdLFszLDIsInBfMSIsMl1d
\[\begin{tikzcd}
	P & B \\
	A & C
	\arrow["{p_2}", from=1-1, to=1-2]
	\arrow["{p_1}"', from=1-1, to=2-1]
	\arrow["g", from=1-2, to=2-2]
	\arrow["f"', from=2-1, to=2-2]
\end{tikzcd}\]


If we need not mention the third coordinate, we may discard it and simply write $(P,f) = p_1 \compl p_2$.

% \section{12 June 2024}

% \begin{mydefinition}
%     Let $\Sigma$ be a signature, \ie a set of function symbols, and $X$ a set of variables disjoint from $\Sigma$. 

%     A hypergraph $G$ is a tuple $(V_G, E_G, s_G, t_G)$ where $V_G$ and $E_G$ are finite sets of nodes (vertices) and hyperedges, $s_G: E_G \rightarrow V_G$ and $t_G: E_G \rightarrow V_G^{*}$ are mappings that assign a source node and a string of target nodes to each edge.
% \end{mydefinition}

% \begin{mydefinition}
%     Let $\Sigma$ be a signature, \ie, a set of function symbols. A (directed) hypergraph $G$ is a tuple $(V_G, E_G, s_G, t_G, l_E)$ where $V_G$ and $E_G$ are finite sets of nodes (vertices) and hyperedges, $s_G: E_G \rightarrow V_G$ and $t_G: E_G \rightarrow V_G^{*}$ are mappings that assign a source node and a string of target nodes to each edge and $l_E : E_G \to \Sigma$ is an edge labelling function. \\

%     We let $\Hyp_{\Sigma}$ be the category of hypergraphs with fixed labels and hypergraph homomorphisms.
% \end{mydefinition}

\begin{mydefinition}%[Alternative definition]
    Let $\Sigma$ be a signature, \ie, a set of function symbols. A (directed) hypergraph $G$ is a tuple $(V_G, E_G, s_G, t_G, l_E)$ where $V_G$ and $E_G$ are finite sets of nodes (vertices) and hyperedges, $s_G, t_G: E_G \rightarrow V_G^{*}$ are mappings that assign a string of source and target nodes to each edge and $l_E : E_G \to \Sigma$ is an edge labelling function. \\

    We let $\Hyp_{\Sigma}$ be the category of hypergraphs with fixed labels and hypergraph homomorphisms.
\end{mydefinition}

\begin{mydefinition}
    A hypergraph with interface is a cospan $\catcospan n G m$ in $\Hyp_{\Sigma}$ where $n$, $m$ are discrete hypergraphs. \dkcom[]{Are f and g monos? In alvarez they're defined as being monos, while in bonchi it is not mentioned. - YES, they are!} \dkcom[]{in bonchi, it is a map J -> G in Hyp, "not necessarily mono even if it is in the majority of the examples}
\end{mydefinition}

To define ma-hypergraphs, we need some new terminology:

\begin{mydefinition}
    The in-degree of a node $v$ in a hypergraph $G$ is the number of pairs $(h, i)$, where $h$ is an hyperedge with $v$ as its $i$-th target.

    Similarly, out-degree of is the number of pairs $(h, j)$ where $h$ is an hyperedge with $v$ as its $j$-th source.
\end{mydefinition}

\dkcom[]{the hypergraph definition should have a list of sources then}
\dkcom[]{bonchi and alvarez allow for multiple inputs, while plump had only multiple outputs.}

\begin{mydefinition}
    A hypergraph $G$ is monogamous acyclic (or an \emph{ma-hypergraph}) if 
    \begin{enumerate}
        \item it contains no cycle, and
        \item every node has at most in- and out- degree 1.
    \end{enumerate}
\end{mydefinition}

In this work we take the approach of \emph{double pushout rewriting} (DPO rewriting) in $\Hyp_{\Sigma}$. A DPO rule is a span $\catspan L K R$ in $\Hyp_{\Sigma}$. A DPO system $\mathcal{R}$ is a finite set of DPO rules. We say that a hypergraph $G$ rewrites into a hypergraph $H$ if there exist a rule $L \xleftarrow{} K \to R$ and an object $C \in \Hyp_{\Sigma}$ such that the following two squares are pushouts: 

% https://q.uiver.app/#q=WzAsNixbMCwxLCJHIl0sWzQsMCwiUiJdLFs0LDEsIkgiXSxbMiwxLCJDIl0sWzIsMCwiSyJdLFswLDAsIkwiXSxbMSwyXSxbMywwXSxbMywyXSxbNCwzXSxbNCwxLCJnIl0sWzQsNSwiZiIsMl0sWzUsMCwibSIsMl0sWzAsMTEsIiIsMix7ImxldmVsIjoxLCJzdHlsZSI6eyJuYW1lIjoiY29ybmVyIn19XSxbMiwxMCwiIiwyLHsibGV2ZWwiOjEsInN0eWxlIjp7Im5hbWUiOiJjb3JuZXIifX1dXQ==
\[\begin{tikzcd}
	L && K && R \\
	G && C && H
	\arrow["m"', from=1-1, to=2-1]
	\arrow[""{name=0, anchor=center, inner sep=0}, "f"', from=1-3, to=1-1]
	\arrow[""{name=1, anchor=center, inner sep=0}, "g", from=1-3, to=1-5]
	\arrow[from=1-3, to=2-3]
	\arrow[from=1-5, to=2-5]
	\arrow[from=2-3, to=2-1]
	\arrow[from=2-3, to=2-5]
	\arrow["\lrcorner"{anchor=center, pos=0.125, rotate=90}, draw=none, from=2-1, to=0]
	\arrow["\lrcorner"{anchor=center, pos=0.125, rotate=180}, draw=none, from=2-5, to=1]
\end{tikzcd}\]

The above works as follows: given a \dkcom[]{This paragraph sounds pretty similar to what is in Bonchi et al., parts 1 and 2. Should we rephrase or cite somehow?} rewrite rule $\catspan L K R$, a hypergraph $G$ and morphism $L \xrightarrow{m} G$, we first compute the pushout complement $C$ which is the part that ``removes'' the matching of $L$ in $G$ while keeping the image of $K$. We then compute the pushout of $\catspan C K R [][g]$ which glues $R$ and $C$ along $K$. 

In what follows, we work with systems in which each rule is \emph{left-linear}. A DPO rule is said to be left-linear if $K \xlongrightarrow{f} L$ is mono. By \cite[Proposition 3.18]{bonchi_rewriting1} if a rewrite rule $K \xlongrightarrow{f}L \xlongrightarrow{m} G$ is left-linear, then the pushout complement is unique, if it exists. 

\vspace{0.5cm}

The pushout complement of a hypergraph exists if and only if the gluing condition is verified.

We remind the definition of the gluing condition, it can be found in \cite[pp.44,45]{ehrig_fundamentals_2006}.
\begin{mydefinition}
    Let $RR = \catspan L K R [f] [g]$ be a rewrite rule and $m : L \to X$ be a matching of the left-hand side of the rule.

    \begin{itemize}
	\item The gluing points $GP$ are the nodes and hyperedges in $L$ that are not deleted by the rewrite rule, namely the nodes and hyperedges in the image of $f$, \ie, $GP = l_V (V_K ) \cup l_E (E_K )$;
	
	\item the identification points $IP$ are the nodes and hyperedges in $L$ that are identified by $m$, namely the nodes and hyperedges with at least two different preimages by $m$, \ie,
	\begin{align*}
		IP &= \{v \in V_L \mid \exists w \in V_L(w) \mathrel. w\neq v \land m_V(v) = m_V(w)\} \\
		&\mathrel{\cup} \{e \in E_L \mid \exists f \in E_L \mathrel. f \neq e \land m_E(e) = m_E(f )\}
	\end{align*}
	
	\item the dangling points $DP$ are the nodes in $L$ whose images under $m$ are the source or target of and edge in $X$ with no preimage by $m$, \ie,
	\[
		DP = \{v \in VL \mid \exists e \in  E_G\compl m_E (E_L) , s_G(e) = m_V (v) \vee t_G(e) = m_V (v)\}
	\]
	\end{itemize}

	$RR$ and $m$ satisfy the gluing condition if $IP \cup DP \subseteq GP$.
    
\end{mydefinition}

For example, let's consider the following rewrite rule with a matching (we labelled the nodes to mention them more easily):

\[
\begin{tikzcd}
		L & K & R \\
        \begin{tikzpicture}[mg,framed]
			\graph {
				1 [mgvertex=left, fill=orange, draw=orange] -> ["a"] 2 [mgvertex=right];
                3 [mgvertex = left, fill=orange, draw=orange] -> ["a"] 4 [mgvertex = right];
                5[mgvertex=left, fill=blue, draw=blue] -!- 6 [mgvertex = right, fill=blue, draw=blue];
            };
		\end{tikzpicture} & \begin{tikzpicture}[mg,framed]
                \graph {
                    1 [mgvertex=left] -!- 2 [mgvertex=right];
                3 [mgvertex = left] -!- ["a"] 4 [mgvertex = right];
                5[mgvertex=left] -!- 6 [mgvertex = right];
                };
        \end{tikzpicture} &
        \begin{tikzpicture}[mg,framed]
			\graph {
				1 [mgvertex=left] -> ["b"] 2 [mgvertex=right];
                3 [mgvertex = left] -> ["c"] 4 [mgvertex = right];
                5[mgvertex=left] -!- 6 [mgvertex = right];
			} ;
		\end{tikzpicture}
        \arrow[from=2-2,to=2-1]
		\arrow[from=2-2,to=2-3]
        \arrow[from=2-1, to=3-1]\\
        \begin{tikzpicture}[mg,framed]
			\graph {
                1 [mgvertex=left] -> ["a"] 2 [mgvertex=right];
                3 [opacity = 0] -!- 4 [mgvertex = right];
                1  -> ["a" {pos=0.5, yshift=-0.15cm}] 4;
                5 [mgvertex=left] -> ["x"] 6 [mgvertex=right];
            };
		\end{tikzpicture}&&
	\end{tikzcd}
\]

The identification points, in orange, are $\{1,3\}$. The dangling points, in blue, are $\{5,6\}$. The gluing points are all the nodes $\{1,2,3,4,5,6\}$. The gluing condition is verified because $IP \cup DP \subseteq GP$ (all orange and blue points have a preimage by $f$).

We can therefore compute the pushout complement:

\[
\begin{tikzcd}
		L & K & R \\
        \begin{tikzpicture}[mg,framed]
			\graph {
				1 [mgvertex=left, fill=orange, draw=orange] -> ["a"] 2 [mgvertex=right];
                3 [mgvertex = left, fill=orange, draw=orange] -> ["a"] 4 [mgvertex = right];
                5[mgvertex=left, fill=blue, draw=blue] -!- 6 [mgvertex = right, fill=blue, draw=blue];
            };
		\end{tikzpicture} & \begin{tikzpicture}[mg,framed]
                \graph {
                    1 [mgvertex=left] -!- 2 [mgvertex=right];
                3 [mgvertex = left] -!- ["a"] 4 [mgvertex = right];
                5[mgvertex=left] -!- 6 [mgvertex = right];
                };
        \end{tikzpicture} &
        \begin{tikzpicture}[mg,framed]
			\graph {
				1 [mgvertex=left] -> ["b"] 2 [mgvertex=right];
                3 [mgvertex = left] -> ["c"] 4 [mgvertex = right];
                5[mgvertex=left] -!- 6 [mgvertex = right];
			} ;
		\end{tikzpicture}
                \arrow[from=2-2,to=2-1]
		\arrow[from=2-2,to=2-3]
        \arrow[from=2-1, to=3-1]
        \arrow[from=2-2, to=3-2]
        \arrow[from=3-2, to=3-1]
            \\
        \begin{tikzpicture}[mg,framed]
			\graph {
                1 [mgvertex=left] -> ["a"] 2 [mgvertex=right];
                3 [opacity = 0] -!- 4 [mgvertex = right];
                1  -> ["a" {pos=0.5, yshift=-0.15cm}] 4;
                5 [mgvertex=left] -> ["x"] 6 [mgvertex=right];
            };
		\end{tikzpicture}&
        \begin{tikzpicture}[mg,framed]
			\graph {
                1 [mgvertex=left] -!- 2 [mgvertex=right];
                3 [opacity = 0] -!- 4 [mgvertex = right];
                5 [mgvertex=left] -> ["x"] 6 [mgvertex=right];
            };
		\end{tikzpicture}&
	\end{tikzcd}
\]

We can see why, if $IP \not \subseteq GP$, the pushout complement would not exist:

\[
\begin{tikzcd}
		L & K & R \\
        \begin{tikzpicture}[mg,framed]
			\graph {
				1 [mgvertex=left, fill=orange, draw=orange] -> ["a", draw=orange] 2 [mgvertex=right, fill=orange, draw=orange];
                3 [mgvertex = left, fill=orange, draw=orange] -> ["a", draw=orange] 4 [mgvertex = right, fill=orange, draw=orange];
                5[mgvertex=left, fill=blue, draw=blue] -!- 6 [mgvertex = right, fill=blue, draw=blue];
            };
		\end{tikzpicture} & \begin{tikzpicture}[mg,framed]
                \graph {
                    1 [mgvertex=left] -!- 2 [mgvertex=right];
                3 [mgvertex = left] -!- ["a"] 4 [mgvertex = right];
                5[mgvertex=left] -!- 6 [mgvertex = right];
                };
        \end{tikzpicture} &
        \begin{tikzpicture}[mg,framed]
			\graph {
				1 [mgvertex=left] -> ["b"] 2 [mgvertex=right];
                3 [mgvertex = left] -> ["c"] 4 [mgvertex = right];
                5[mgvertex=left] -!- 6 [mgvertex = right];
			} ;
		\end{tikzpicture}
        \arrow[from=2-2,to=2-1]
		\arrow[from=2-2,to=2-3]
        \arrow[from=2-1, to=3-1]
        \arrow[from=2-2, to=3-2]
        \arrow[from=3-2, to=3-1]\\
        \begin{tikzpicture}[mg,framed]
			\graph {
                1 [mgvertex=left] -> ["a"] 2 [mgvertex=right];
                5 [mgvertex=left] -> ["x"] 6 [mgvertex=right];
            };
		\end{tikzpicture}&\times&
	\end{tikzcd}
\]

No hypergraph  can make the left square a pushout: we cannot remove two $a$ hyperedges in the matching because there is only one $a$ hyperedge in the matching. The ``best'' we can do is the following construction (note that the pushout is not the one we wanted):

\[
\begin{tikzcd}
		L & K & R \\
        \begin{tikzpicture}[mg,framed]
			\graph {
				1 [mgvertex=left, fill=orange, draw=orange] -> ["a", draw=orange] 2 [mgvertex=right, fill=orange, draw=orange];
                3 [mgvertex = left, fill=orange, draw=orange] -> ["a", draw=orange] 4 [mgvertex = right, fill=orange, draw=orange];
                5[mgvertex=left, fill=blue, draw=blue] -!- 6 [mgvertex = right, fill=blue, draw=blue];
            };
		\end{tikzpicture} & \begin{tikzpicture}[mg,framed]
                \graph {
                    1 [mgvertex=left] -!- 2 [mgvertex=right];
                3 [mgvertex = left] -!- ["a"] 4 [mgvertex = right];
                5[mgvertex=left] -!- 6 [mgvertex = right];
                };
        \end{tikzpicture} &
        \begin{tikzpicture}[mg,framed]
			\graph {
				1 [mgvertex=left] -> ["b"] 2 [mgvertex=right];
                3 [mgvertex = left] -> ["c"] 4 [mgvertex = right];
                5[mgvertex=left] -!- 6 [mgvertex = right];
			} ;
		\end{tikzpicture}
        \arrow[from=2-2,to=2-1]
		\arrow[from=2-2,to=2-3]
        \arrow[from=2-1, to=3-1]
        \arrow[from=2-2, to=3-2]
        \arrow[from=3-2, to=3-1]\\
        \begin{tikzpicture}[mg,framed]
			\graph {
                1 [mgvertex=left] -> ["a", bend left] 2 [mgvertex=right];
                1 -> ["a", bend right] 2;
                5 [mgvertex=left] -> ["x"] 6 [mgvertex=right];
            };
		\end{tikzpicture}&
        \begin{tikzpicture}[mg,framed]
			\graph {
                1 [mgvertex=left] -!- 2 [mgvertex=right];
                5 [mgvertex=left] -> ["x"] 6 [mgvertex=right];
            };
		\end{tikzpicture}&
	\end{tikzcd}
\]



We can also see why if $DP \not \subseteq GP$, the pushout complement would not exist~:

\[
\begin{tikzcd}
		L & K & R \\
        \begin{tikzpicture}[mg,framed]
			\graph {
				1 [mgvertex=left, fill=orange, draw=orange] -> ["a"] 2 [mgvertex=right];
                3 [mgvertex = left, fill=orange, draw=orange] -> ["a"] 4 [mgvertex = right];
                5[mgvertex=left, fill=blue, draw=blue] -!- 6 [mgvertex = right, fill=blue, draw=blue];
            };
		\end{tikzpicture} & \begin{tikzpicture}[mg,framed]
                \graph {
                    1 [mgvertex=left] -!- 2 [mgvertex=right];
                3 [mgvertex = left] -!- ["a"] 4 [mgvertex = right];
                };
        \end{tikzpicture} &
        \begin{tikzpicture}[mg,framed]
			\graph {
				1 [mgvertex=left] -> ["b"] 2 [mgvertex=right];
                3 [mgvertex = left] -> ["c"] 4 [mgvertex = right];
			} ;
		\end{tikzpicture}
                \arrow[from=2-2,to=2-1]
		\arrow[from=2-2,to=2-3]
        \arrow[from=2-1, to=3-1]
        \arrow[from=2-2, to=3-2]
        \arrow[from=3-2, to=3-1]
            \\
        \begin{tikzpicture}[mg,framed]
			\graph {
                1 [mgvertex=left] -> ["a"] 2 [mgvertex=right];
                3 [opacity = 0] -!- 4 [mgvertex = right];
                1  -> ["a" {pos=0.5, yshift=-0.15cm}] 4;
                5 [mgvertex=left] -> ["x"] 6 [mgvertex=right];
            };
		\end{tikzpicture}&
        \times&
	\end{tikzcd}
\]

We cannot remove the nodes $5$ and $6$ from the matching without removing the hyperedge $x$, the hyperedge $x$ would be left with a source and a target missing.



\begin{mydefinition}[Pre-critical pair] \label{dfn:pcPair}
    Let $\mathcal{R}$ be a DPO system with rules $\catspan {L_1} {K_1} {R_1}$ and $\catspan {L_2} {K_2} {R_2}$. Consider two derivations with common source $S$:

    % https://q.uiver.app/#q=WzAsMTEsWzAsMCwiUl8xIl0sWzEsMCwiS18xIl0sWzIsMCwiTF8xIl0sWzQsMCwiTF8yIl0sWzUsMCwiS18yIl0sWzMsMSwiUyJdLFswLDEsIkhfMSJdLFsxLDEsIkNfMSJdLFs1LDEsIkNfMiJdLFs2LDEsIkhfMiJdLFs2LDAsIlJfMiJdLFsxLDJdLFsxLDBdLFs0LDNdLFs0LDEwXSxbMyw1LCJmXzIiLDJdLFsyLDUsImZfMSJdLFs3LDZdLFswLDZdLFsxLDddLFs4LDVdLFs4LDldLFsxMCw5XSxbNCw4XSxbNiwxLCIiLDEseyJzdHlsZSI6eyJuYW1lIjoiY29ybmVyIn19XSxbNSwxLCIiLDEseyJzdHlsZSI6eyJuYW1lIjoiY29ybmVyIn19XSxbNSw0LCIiLDEseyJzdHlsZSI6eyJuYW1lIjoiY29ybmVyIn19XSxbNyw1LCIiLDIseyJsYWJlbF9wb3NpdGlvbiI6NDB9XSxbOSw0LCIiLDIseyJzdHlsZSI6eyJuYW1lIjoiY29ybmVyIn19XV0=
\[\begin{tikzcd}
	{R_1} & {K_1} & {L_1} && {L_2} & {K_2} & {R_2} \\
	{H_1} & {C_1} && S && {C_2} & {H_2}
	\arrow[from=1-1, to=2-1]
	\arrow[from=1-2, to=1-1]
	\arrow[from=1-2, to=1-3]
	\arrow[from=1-2, to=2-2]
	\arrow["{m_1}", from=1-3, to=2-4]
	\arrow["{m_2}"', from=1-5, to=2-4]
	\arrow[from=1-6, to=1-5]
	\arrow[from=1-6, to=1-7]
	\arrow[from=1-6, to=2-6]
	\arrow[from=1-7, to=2-7]
	\arrow["\lrcorner"{anchor=center, pos=0.125, rotate=90}, draw=none, from=2-1, to=1-2]
	\arrow[from=2-2, to=2-1]
	\arrow[from=2-2, to=2-4]
	\arrow["\lrcorner"{anchor=center, pos=0.125, rotate=180}, draw=none, from=2-4, to=1-2]
	\arrow["\lrcorner"{anchor=center, pos=0.125, rotate=90}, draw=none, from=2-4, to=1-6]
	\arrow[from=2-6, to=2-4]
	\arrow[from=2-6, to=2-7]
	\arrow["\lrcorner"{anchor=center, pos=0.125, rotate=180}, draw=none, from=2-7, to=1-6]
\end{tikzcd}\]

We say that $H_1 \leftsquigarrow S \rightsquigarrow H_2$ is a pre-critical pair if $[m_1, m_2] : L_1 + L_2 \to S$ is epi; it is joinable if there
\end{mydefinition}
exists $W$ such that $H_1 \stackrel{*}{\rightsquigarrow} W \stackrel{*}{\leftsquigarrow} H_2$.

\begin{mydefinition} \label{def:parallel_pair}
    Let $\mathcal{R}$ be a DPO system with rules $\catspan {L_1} {K_1} {R_1}$ and $\catspan {L_2} {K_2} {R_2}$ .
Consider two derivations with common source $S$ as in Definition \ref{dfn:pcPair}. We say that $H_1 \leftsquigarrow S \rightsquigarrow H_2$ is a parallel pair if there exist $g_1 : L_1 \to C_2$ and $g_2 : L_2 \to C_1$ making the diagram below commute:

\[\begin{tikzcd}[ampersand replacement=\&, column sep=small]
	{K_1} \& {L_1} \&\& {L_2} \& {K_2} \\
	{C_1} \&\& S \&\& {C_2}
	\arrow["{f_1}", hook, from=1-1, to=1-2]
	\arrow[from=1-1, to=2-1]
	\arrow[from=1-2, to=2-3]
	\arrow[from=1-2, to=2-5]
	\arrow[from=1-4, to=2-1]
	\arrow[from=1-4, to=2-3]
	\arrow["{f_2}"', hook', from=1-5, to=1-4]
	\arrow[from=1-5, to=2-5]
	\arrow[from=2-1, to=2-3]
    \arrow[from=2-5, to=2-3]
\end{tikzcd}\]

\end{mydefinition}



Let's give an example, consider the following rewrite system:

\begin{myproposition}
    Let $R$ be the output of Algo.~\ref{algo1}. Then:
    \begin{description}
        \item[Correctness] For each $(\epsilon,F) \in R$, $\epsilon$ is an epimorphism.
        \item[Exhaustiveness] For any critical pair $L_1 + L_2 \overset{\epsilon}{\twoheadrightarrow} X \overset{F}{\leftarrow} J$, we have $(\epsilon,F) \in R$.
        \item[Effectiveness] $R$ contains critical pairs.
    \end{description}
\end{myproposition}

% https://tikzcd.yichuanshen.de/#N4Igdg9gJgpgziAXAbVABwnAlgFyxMJZABgBpiBdUkANwEMAbAVxiRAHEQBfU9TXfIRRkAjFVqMWbADIB9YCK4ACANRK5wAExduvEBmx4CRMpvH1mrRCACinHn0OCTpAMznJVkAA1u4mFAA5vBEoABmAE4QALZIZCA4EEgi1ABGMGBQSAC0rsQOIJExcdSJyWkZWYh5BUWxiCkJSYia1AxYYF5QEDg4ASDUABYwdFUJAO4Qw6MIbXTpDAAK-EZCIBFYgYM4AxKWbACOuuFR9a1NSK4VmTl5cwvLTsbWG1s71BZS1gAex4WnSHOZWq1yqdxADHmMCWK2c1gYMDC7z2XxAAE8-FwgA
\[\begin{tikzcd}
G \arrow[d, bend right] \arrow[d, bend left]             \\
L_{1} + L_{2} \arrow[d, "q"', two heads, dotted]         \\
EG \arrow[d, "x"', bend right] \arrow[d, "y", bend left] \\
X                                                       
\end{tikzcd}
\qquad
% https://tikzcd.yichuanshen.de/#N4Igdg9gJgpgziAXAbVABwnAlgFyxMJZABgBpiBdUkANwEMAbAVxiRAHEQBfU9TXfIRRkAjFVqMWbADIB9YCK4ACANRK5wAExduvEBmx4CRMpvH1mrRCACinHn0OCiI0meoWp1gBrdxMKABzeCJQADMAJwgAWyQyEBwIJFcQACMYMCgkAFoAZnjPKxAwkGoGOnSGAAV+IyEQCKxAgAscXXCo2MR4xOTqdMykfI9JIsD24s6+hKTETTKsMCKoCBwcANKQZpg6LOscAHcIbd2EMoqYatrna0aWtpHLNgBHCciYpHmZocevEAAPAC8AE9NuVKjUnMZbk1Wm8pogUr1ELlfkVnv94R9utRkaiJE9rGEMUpAUpAhi-FwgA
\begin{tikzcd}
G \arrow[d, "f"', bend right] \arrow[d, "g", bend left] \arrow[rdd, bend left, "fqx = gqx"] &   \\
L_{1} + L_{2} \arrow[d, "q"', two heads, dotted] \arrow[rd, "qx"]                &   \\
EG \arrow[r, "x=y"']                                                             & X
\end{tikzcd}\]

% \section{13 June 2024}

\begin{mydefinition}[Pre-critical pair with interface {\cite[Def.~3.1]{DBLP:journals/mscs/BonchiGKSZ22a}}]
    Let $\mathcal{R}$ be a DPO system with rules $\catspan {L_i} {K_i} {R_i} [f_i] [g_i]$ where $i \in \{1,2\}$ are two rewrite rules. Consider two derivations with source $S \leftarrow J$

% https://q.uiver.app/#q=WzAsMTIsWzAsMCwiUl8xIl0sWzEsMCwiS18xIl0sWzIsMCwiTF8xIl0sWzQsMCwiTF8yIl0sWzUsMCwiS18yIl0sWzYsMCwiUl8yIl0sWzAsMSwiSF8xIl0sWzEsMSwiQ18xIl0sWzMsMSwiUyJdLFs1LDEsIkNfMiJdLFs2LDEsIkhfMiJdLFszLDIsIkoiXSxbMCw2XSxbMSwwXSxbMSwyXSxbMSw3XSxbMiw4LCJ7Zl8xfSJdLFszLDgsIntmXzJ9IiwyXSxbNCwzXSxbNCw1XSxbNCw5XSxbNSwxMF0sWzYsMSwiIiwwLHsic3R5bGUiOnsibmFtZSI6ImNvcm5lciIsImJvZHkiOnsibmFtZSI6Im5vbmUifSwiaGVhZCI6eyJuYW1lIjoibm9uZSJ9fX1dLFs3LDZdLFs3LDhdLFs4LDEsIiIsMCx7InN0eWxlIjp7Im5hbWUiOiJjb3JuZXIiLCJib2R5Ijp7Im5hbWUiOiJub25lIn0sImhlYWQiOnsibmFtZSI6Im5vbmUifX19XSxbOCw0LCIiLDAseyJzdHlsZSI6eyJuYW1lIjoiY29ybmVyIiwiYm9keSI6eyJuYW1lIjoibm9uZSJ9LCJoZWFkIjp7Im5hbWUiOiJub25lIn19fV0sWzksOF0sWzksMTBdLFsxMCw0LCIiLDAseyJzdHlsZSI6eyJuYW1lIjoiY29ybmVyIiwiYm9keSI6eyJuYW1lIjoibm9uZSJ9LCJoZWFkIjp7Im5hbWUiOiJub25lIn19fV0sWzExLDksIiIsMCx7ImN1cnZlIjoxfV0sWzExLDcsIiIsMCx7ImN1cnZlIjotMX1dLFsxMSw4LCIiLDAseyJjb2xvdXIiOlsxMjAsNjAsNjBdLCJzdHlsZSI6eyJuYW1lIjoiYWRqdW5jdGlvbiJ9fV1d
\[\begin{tikzcd}[ampersand replacement=\&]
	{R_1} \& {K_1} \& {L_1} \&\& {L_2} \& {K_2} \& {R_2} \\
	{H_1} \& {C_1} \&\& S \&\& {C_2} \& {H_2} \\
	\&\&\& J
	\arrow[from=1-1, to=2-1]
	\arrow[from=1-2, to=1-1]
	\arrow[from=1-2, to=1-3]
	\arrow[from=1-2, to=2-2]
	\arrow["{{m_1}}", from=1-3, to=2-4]
	\arrow["{{m_2}}"', from=1-5, to=2-4]
	\arrow[from=1-6, to=1-5]
	\arrow[from=1-6, to=1-7]
	\arrow[from=1-6, to=2-6]
	\arrow[from=1-7, to=2-7]
	\arrow["\lrcorner"{anchor=center, pos=0.125, rotate=90}, draw=none, from=2-1, to=1-2]
	\arrow[from=2-2, to=2-1]
	\arrow[from=2-2, to=2-4]
	\arrow["\lrcorner"{anchor=center, pos=0.125, rotate=180}, draw=none, from=2-4, to=1-2]
	\arrow["\lrcorner"{anchor=center, pos=0.125, rotate=90}, draw=none, from=2-4, to=1-6]
	\arrow[from=2-6, to=2-4]
	\arrow[from=2-6, to=2-7]
	\arrow["\lrcorner"{anchor=center, pos=0.125, rotate=180}, draw=none, from=2-7, to=1-6]
	\arrow[curve={height=-6pt}, from=3-4, to=2-2]
	\arrow["\dashv"{anchor=center, rotate=90}, color={rgb,255:red,92;green,214;blue,92}, draw=none, from=3-4, to=2-4]
	\arrow[curve={height=6pt}, from=3-4, to=2-6]
\end{tikzcd}\]

We say that $(H_1 \leftarrow J) \leftsquigarrow (S \leftarrow J) \rightsquigarrow (H_2 \leftarrow J)$ is a pre-critical pair if $[m_1, m_2] : L_1 + L_2 \rightarrow S$ is epi and {\color{green}T} is a pullback; it is joinable if there exists $W \leftarrow J$ such that $(H_1 \leftarrow J) \stackrel{*}{\rightsquigarrow} (W \leftarrow J) \stackrel{*}{\leftsquigarrow} (H_2 \leftarrow J)$.
\end{mydefinition}

\begin{algorithm}[H]
    \caption{An algorithm for enumerating pre-critical pairs}\label{algo2}
    \SetKwInOut{Input}{Input}\SetKwInOut{Output}{Output}\SetKw{Yield}{yield}
    \Input{rewrite rules $\rho = \{ L_i \overset{f_i}{\leftarrow} K_i \overset{g_i}{\rightarrow} R_i \}_{i \in I}$}
    \Output{epimorphisms with interface $\{ \{ L_i + L_j \overset{\epsilon}{\twoheadrightarrow} S_{ij\gamma} \leftarrow J_{ij\gamma} \}_{\gamma \in I_{ij}} \}_{(i,j) \in I^2}$}
    \For{$(i,j) \in I^2$}{
        $E_s = \mathcal{P}((\mathrm{label};p_1) \times_{\mathcal{E}_{L_i \times L_j}} (\mathrm{label};p_2))$\;
        \tcc{subsets of the equalizer of $\Bigl( % https://q.uiver.app/#q=WzAsMixbMCwwLCJcXG1hdGhjYWx7RX1fe0xfaSBcXHRpbWVzIExfan0iXSxbMiwwLCJcXFNpZ21hIl0sWzAsMSwibGFiZWwgOyBwXzIiLDIseyJjdXJ2ZSI6MX1dLFswLDEsImxhYmVsIDsgcF8xIiwwLHsiY3VydmUiOi0xfV1d
\begin{tikzcd}[ampersand replacement=\&,cramped]
	{\mathcal{E}_{L_i \times L_j}} \&\& \Sigma
	\arrow["{label ; p_2}"', curve={height=6pt}, from=1-1, to=1-3]
	\arrow["{label ; p_1}", curve={height=-6pt}, from=1-1, to=1-3]
\end{tikzcd} \Bigr)$ in $\mathbf{Set}$ (edges of $L_i\times L_j$ with the same label)} \label{algo2:line}
        $V_s = \mathcal{P}(\mathcal{V}_{L_{i} \times L_{j}})$\;
        $\mathcal{G} = \lbrace \langle e, v\rangle_{L_{i} \times L_{j}} \ | \ e \in E_s, \ v \in V_s \rbrace$\;
        %\tcc{induced subgraphs}
        \tcc{$\langle e,v \rangle$ is the smallest graph containing the edges $e$ and the vertices $v$ (we may have to add vertices to $v$ to account for sources and targets of edges not present in $v$).}
        \For{$\gamma \leftarrow \mathcal{G}$}{
            $(S_{ij\gamma}, \epsilon_{ij\gamma}) = \mathtt{coeq}_{\gamma}(p_1;q_1, p_2;q_2)$\;
            \tcc{the coequalizer of (% https://q.uiver.app/#q=WzAsMyxbMCwwLCJcXGdhbW1hIl0sWzEsMCwiTF8xK0xfMiJdLFsyLDAsIlNfe2lqXFxnYW1tYX0iXSxbMCwxLCJwXzIgOyBxXzIiLDIseyJjdXJ2ZSI6MX1dLFswLDEsInBfMSA7IHFfMSIsMCx7ImN1cnZlIjotMX1dLFsxLDIsIlxcZXBzaWxvbl97aWpcXGdhbW1hfSIsMCx7InN0eWxlIjp7ImJvZHkiOnsibmFtZSI6ImRhc2hlZCJ9LCJoZWFkIjp7Im5hbWUiOiJlcGkifX19XV0=
\begin{tikzcd}[ampersand replacement=\&]
	\gamma \& {L_i+L_j} \& {S_{ij\gamma}}
	\arrow["{p_2 ; q_2}", near end, bend left = 15, from=1-1, to=1-2]
	\arrow["{p_1 ; q_1}"', near end, bend right = 15, from=1-1, to=1-2]
	\arrow["{\epsilon_{ij\gamma}}", dashed, two heads, from=1-2, to=1-3]
\end{tikzcd}) in $\mathbf{Hyp}_\Sigma$}
            $(C_{ij\gamma}^L,c_{ij\gamma}^L) = (q_1;\epsilon_{ij\gamma}) \compl f_i$\;
            \tcc{the pushout complement of $\Bigl( % https://q.uiver.app/#q=WzAsNCxbMCwwLCJMX2oiXSxbMSwwLCJLX2oiXSxbMCwxLCJTX3tpalxcZ2FtbWF9Il0sWzEsMSwiQ197bGpcXGdhbW1hfV5SIl0sWzAsMiwie3FfMTtcXGVwc2lsb25fe2lqXFxnYW1tYX19IiwyXSxbMSwwLCJ7Zl9qfSIsMl0sWzEsM10sWzIsMSwiIiwwLHsic3R5bGUiOnsibmFtZSI6ImNvcm5lciIsImJvZHkiOnsibmFtZSI6Im5vbmUifSwiaGVhZCI6eyJuYW1lIjoibm9uZSJ9fX1dLFszLDIsIntjX3tsalxcZ2FtbWF9XlJ9Il1d
\begin{tikzcd}[ampersand replacement=\&]
	{L_i} \& {K_i} \\
	{S_{ij\gamma}} \& {C_{ij\gamma}^R}
	\arrow["{{q_1;\epsilon_{ij\gamma}}}"', from=1-1, to=2-1]
	\arrow["{{f_i}}"', from=1-2, to=1-1]
	\arrow[from=1-2, to=2-2]
	\arrow["\lrcorner"{anchor=center, pos=0.125, rotate=90}, draw=none, from=2-1, to=1-2]
	\arrow["{{c_{ij\gamma}^L}}", from=2-2, to=2-1]
\end{tikzcd} \Bigr)$ in $\mathbf{Hyp}_\Sigma$}
            $(C_{ij\gamma}^R,c_{ij\gamma}^R) = (q_2;\epsilon_{ij\gamma}) \compl f_j$\;
            \tcc{the pushout complement of $\Bigl(  % https://q.uiver.app/#q=WzAsNCxbMCwwLCJMX2oiXSxbMCwxLCJTX3tpalxcZ2FtbWF9Il0sWzEsMSwiQ197bGpcXGdhbW1hfV5SIl0sWzEsMCwiS19qIl0sWzMsMCwiZl9qIiwyXSxbMCwxLCJxXzI7XFxlcHNpbG9uX3tpalxcZ2FtbWF9IiwyXSxbMiwxLCJjX3tsalxcZ2FtbWF9XlIiXSxbMywyXSxbMSwzLCIiLDEseyJzdHlsZSI6eyJuYW1lIjoiY29ybmVyIn19XV0=
\begin{tikzcd}[ampersand replacement=\&]
	{L_j} \& {K_j} \\
	{S_{ij\gamma}} \& {C_{ij\gamma}^R}
	\arrow["{q_2;\epsilon_{ij\gamma}}"', from=1-1, to=2-1]
	\arrow["{f_j}"', from=1-2, to=1-1]
	\arrow[from=1-2, to=2-2]
	\arrow["\lrcorner"{anchor=center, pos=0.125, rotate=90}, draw=none, from=2-1, to=1-2]
	\arrow["{c_{ij\gamma}^R}", from=2-2, to=2-1]
\end{tikzcd}\Bigr)$ in $\mathbf{Hyp}_\Sigma$}
            $J_{ij\gamma} = c_{ij\gamma}^L \times_{S_{ij\gamma}} c_{ij\gamma}^R$\;
            \tcc{the pullback of $\Bigl(\begin{tikzcd}[ampersand replacement=\&]
                    	J_{ij\gamma} \& {C^R_{ij\gamma}} \\
                    	{C^L_{ij\gamma}} \& S_{ij\gamma}
                    	\arrow["{\pi_2}", from=1-1, to=1-2]
                    	\arrow["{\pi_1}"', from=1-1, to=2-1]
                    	\arrow["\ulcorner"{anchor=center, pos=0.125}, draw=none, from=1-1, to=2-2]
                    	\arrow["{c_{ij\gamma}^R}", from=1-2, to=2-2]
                    	\arrow["{c_{ij\gamma}^L}"', from=2-1, to=2-2]
                    \end{tikzcd}\Bigr)$ in $\mathbf{Hyp}_\Sigma$}
            \Yield{$L_i + L_j \overset{\epsilon_{ij\gamma}}{\twoheadrightarrow} S_{ij\gamma} \xleftarrow{\pi_1;c_{ij\gamma}^L} J_{ij\gamma}$}\;
        }
    }
\end{algorithm}

\begin{mylemma}
    Let $L_1 \xleftarrow{f_1} K_1 \xrightarrow[]{} R_1$ and $L_2 \xleftarrow{f_2} K_2 \xrightarrow[]{} R_2$ be two rewrites. Then the coproduct $L_1 + L_2$ is a parallel pre-critical pair. 
\end{mylemma}
\begin{myproof}
    Let $(L_1 + L_2, i_1, i_2)$ be the coproduct of $L_1$ and $L_2$.\\
    We start by showing that $(L_1 + L_2, i_1, [f_1,i_2])$ is the pushout of $K_1 + L_2 \xleftarrow{\imath_{K_1}} K_1 \xrightarrow{f_1} L_1$. It is clear that $q_1$ and $[f_1, i_2]$ make the diagram commute. Now we need to show that $(L_1 + L_2, i_1, [f_1,i_2])$ has the universal property.\\
    Since we are working with left-linear system, $f_1$ is a monomorphism and $K_1 + L_2$ is the unique pushout complement that makes the diagram $[K_1, f_1, \imath_{K_1}]$ a pushout diagram. (A similar argument applies to the right side of the diagram).
    % https://q.uiver.app/#q=WzAsMTEsWzYsMCwiS18yIl0sWzUsMCwiTF8yIl0sWzIsMCwiS18xIl0sWzMsMCwiTF8xIl0sWzQsMSwiTF8xICsgTF8yIl0sWzYsMSwiTF8xICsgS18yIl0sWzIsMSwiS18xICsgTF8yIl0sWzAsMCwiUl8xIl0sWzAsMSwiUl8xICsgTF8yIl0sWzgsMCwiUl8yIl0sWzgsMSwiTF8xICsgUl8yIl0sWzAsMSwiZl8yIiwyXSxbMiwzLCJmXzEiXSxbMyw0LCJpXzEiXSxbMSw0LCJpXzIiLDJdLFswLDUsIlxcaW1hdGhfe0tfMn0iXSxbNSw0LCJbaV8yLGZfMl0iXSxbMiw2LCJcXGltYXRoX3tLXzF9IiwyXSxbNiw0LCJbZl8xLGlfMl0iLDJdLFsyLDcsImdfMSIsMl0sWzcsOF0sWzYsOCwiW2dfMSxcXHRleHR7aWR9X3tMXzJ9XSJdLFswLDksImdfMiJdLFs1LDEwLCJbXFx0ZXh0e2lkfV97TF8xfSxnXzJdIiwyXSxbOSwxMF1d
\[\begin{tikzcd}
	{R_1} && {K_1} & {L_1} && {L_2} & {K_2} && {R_2} \\
	{R_1 + L_2} && {K_1 + L_2} && {L_1 + L_2} && {L_1 + K_2} && {L_1 + R_2}
	\arrow[from=1-1, to=2-1]
	\arrow["{g_1}"', from=1-3, to=1-1]
	\arrow["{f_1}", from=1-3, to=1-4]
	\arrow["{\imath_{K_1}}"', from=1-3, to=2-3]
	\arrow["{i_1}", from=1-4, to=2-5]
	\arrow["{i_2}"', from=1-6, to=2-5]
	\arrow["{f_2}"', from=1-7, to=1-6]
	\arrow["{g_2}", from=1-7, to=1-9]
	\arrow["{\imath_{K_2}}", from=1-7, to=2-7]
	\arrow[from=1-9, to=2-9]
	\arrow["{[g_1,\text{id}_{L_2}]}", from=2-3, to=2-1]
	\arrow["{[f_1,i_2]}"', from=2-3, to=2-5]
	\arrow["{[i_2,f_2]}", from=2-7, to=2-5]
	\arrow["{[\text{id}_{L_1},g_2]}"', from=2-7, to=2-9]
\end{tikzcd}\]


Consider the inclusions $\imath_{L_1} : L_1 \to L_1 + K_2$ and $\imath_{L_2} : L_2 \to K_1 + L_2$; we have that $\imath_{L_1} ; [f_1, \id] = \imath_{L_2} ; [\id, f_2]$. We conclude that $L_1 + L_2$ is a parallel critical pair since the identity is an epimorphism.

% https://q.uiver.app/#q=WzAsOCxbNCwwLCJLXzIiXSxbMywwLCJMXzIiXSxbMCwwLCJLXzEiXSxbMSwwLCJMXzEiXSxbMiwxLCJMXzEgKyBMXzIiXSxbNCwxLCJMXzEgKyBLXzIiXSxbMCwxLCJLXzEgKyBMXzIiXSxbMSwxXSxbMCwxLCJmXzIiLDJdLFsyLDMsImZfMSJdLFszLDQsInFfMSJdLFsxLDQsInFfMiIsMl0sWzAsNSwiXFxpbWF0aF97S18yfSJdLFs1LDQsIltcXHRleHR7aWR9X3tMXzF9LGZfMl0iXSxbMiw2LCJcXGltYXRoX3tLXzF9IiwyXSxbNiw0LCJbZl8xLFxcdGV4dHtpZH1fe0xfMn1dIiwyXSxbMyw1LCIiLDEseyJzdHlsZSI6eyJ0YWlsIjp7Im5hbWUiOiJob29rIiwic2lkZSI6InRvcCJ9fX1dLFsxLDYsIiIsMSx7InN0eWxlIjp7InRhaWwiOnsibmFtZSI6Imhvb2siLCJzaWRlIjoiYm90dG9tIn19fV1d
\[\begin{tikzcd}
	{K_1} & {L_1} && {L_2} & {K_2} \\
	{K_1 + L_2} & {} & {L_1 + L_2} && {L_1 + K_2}
	\arrow["{f_1}", from=1-1, to=1-2]
	\arrow["{\imath_{K_1}}"', from=1-1, to=2-1]
	\arrow["{q_1}", from=1-2, to=2-3]
	\arrow[hook, from=1-2, to=2-5]
	\arrow[hook', from=1-4, to=2-1]
	\arrow["{q_2}"', from=1-4, to=2-3]
	\arrow["{f_2}"', from=1-5, to=1-4]
	\arrow["{\imath_{K_2}}", from=1-5, to=2-5]
	\arrow["{[f_1,\text{id}_{L_2}]}"', from=2-1, to=2-3]
	\arrow["{[\text{id}_{L_1},f_2]}", from=2-5, to=2-3]
\end{tikzcd}\]

\end{myproof}


\begin{myproposition}
    \noindent
    \begin{description}
        \item[Correctness] \label{item:correctness} Each result $L_i + L_j \overset{\epsilon_{ij\gamma}}{\twoheadrightarrow} S_{ij\gamma} \xleftarrow{\pi_1;c_{ij\gamma}^L} J_{ij\gamma}$ of Algo.~\ref{algo2} is a pre-critical pair.
        \item[Exhaustiveness] \label{item:exhaustiveness} Any pre-critical pair $L_i + L_j \overset{\epsilon}{\twoheadrightarrow} X \overset{F}{\leftarrow} J$ can be yielded by Algo.~\ref{algo2}.
    \end{description}
\end{myproposition}
\begin{myproof}[Proof of correctness]
   A coequalizer is an epimorphism, thus $\epsilon_{ij\gamma}$ is epi. $J_{ij\gamma}$ is the pullback required by the pre-critical pair with interface definition, by construction.

   Therefore, each result yielded is a valid precritical-pair with interface.
   
\end{myproof}
\begin{myproof}[Proof of exhaustiveness]
Let $L_i + L_j \overset{\epsilon}{\twoheadrightarrow} X \overset{F}{\leftarrow} J$ be a precritical-pair with interface.

An epimorphism of hypergraphs is surjective on nodes and on hyperedges because the category of hypergraphs is a presheaf category.

Thus, each node and each hyperedge has a non-empty preimage set by $\epsilon$. We can see each of these preimages sets as equivalence classes glued by a coequalizer process, as will be explained by what follows.

The inclusions $\iota_i;\epsilon$ and $\iota_j;\epsilon$ should be monomorphisms for the complements $C_i$ and $C_j$ to exist.


problem : there are not necessarily monomorphisms. There are only injective on hyperedges.
for example, you could have L1 = 1 -f-> 2 3 -g-> 4 and in X have 1 -f-> 2 -g-> 4.
% We must only consider ma-cospans (ma hypergraphs with interface) for it to be true (Definition 31 Bonchi III, follow section 5 of Bonchi III)

% https://q.uiver.app/#q=WzAsOSxbMSwwLCJMX2kiXSxbMywwLCJMX2oiXSxbMiwxLCJMX2krTF9qIl0sWzIsMiwiWCJdLFsyLDMsIkoiXSxbMCwyLCJDX2kiXSxbNCwyLCJDX2oiXSxbMCwwLCJLX2kiXSxbNCwwLCJLX2oiXSxbMCwyLCJcXGlvdGFfaSJdLFsxLDIsIlxcaW90YV9qIiwyXSxbMiwzLCJcXGVwc2lsb24iLDEseyJzdHlsZSI6eyJoZWFkIjp7Im5hbWUiOiJlcGkifX19XSxbMCwzLCJcXGlvdGFfaTtcXGVwc2lsb24iLDJdLFsxLDMsIlxcaW90YV9qO1xcZXBzaWxvbiJdLFs0LDVdLFs0LDZdLFs1LDNdLFs2LDNdLFs3LDBdLFs4LDFdLFs3LDVdLFs4LDZdLFszLDcsIiIsMCx7InN0eWxlIjp7Im5hbWUiOiJjb3JuZXIifX1dLFszLDgsIiIsMCx7InN0eWxlIjp7Im5hbWUiOiJjb3JuZXIifX1dLFs0LDMsIiIsMCx7InN0eWxlIjp7Im5hbWUiOiJjb3JuZXIifX1dXQ==
\[\begin{tikzcd}[ampersand replacement=\&,cramped]
	{K_i} \& {L_i} \&\& {L_j} \& {K_j} \\
	\&\& {L_i+L_j} \\
	{C_i} \&\& X \&\& {C_j} \\
	\&\& J
	\arrow[from=1-1, to=1-2]
	\arrow[from=1-1, to=3-1]
	\arrow["{\iota_i}", from=1-2, to=2-3]
	\arrow["{\iota_i;\epsilon}"', from=1-2, to=3-3]
	\arrow["{\iota_j}"', from=1-4, to=2-3]
	\arrow["{\iota_j;\epsilon}", from=1-4, to=3-3]
	\arrow[from=1-5, to=1-4]
	\arrow[from=1-5, to=3-5]
	\arrow["\epsilon"{description}, two heads, from=2-3, to=3-3]
	\arrow[from=3-1, to=3-3]
	\arrow["\lrcorner"{anchor=center, pos=0.125, rotate=180}, draw=none, from=3-3, to=1-1]
	\arrow["\lrcorner"{anchor=center, pos=0.125, rotate=90}, draw=none, from=3-3, to=1-5]
	\arrow[from=3-5, to=3-3]
	\arrow[from=4-3, to=3-1]
	\arrow["\lrcorner"{anchor=center, pos=0.125, rotate=135}, draw=none, from=4-3, to=3-3]
	\arrow[from=4-3, to=3-5]
\end{tikzcd}\]

Therefore, two nodes in a preimage equivalence class cannot come from the same hypergraph and two hyperedges in a preimage equivalence class cannot come from the same hypergraph. The preimage equivalence classes can therefore have at most two elements each.

We construct the hypergraph $\gamma$ with the preimage equivalence classes of nodes as nodes and the preimage equivalence classes of hyperedges as hyperedges. We consider two hypergraph homomorphisms $F_1 : \gamma \to L_1 + L_2$ and $F_2 : \gamma \to L_1 + L_2$ such that $F_1$ maps an equivalence class node (hyperedge) to the node (hyperedge) coming from $L_1$ if possible and $F_2$ maps an equivalence class node (hyperedge) to the node (hyperedge) coming from $L_2$ if possible.

$X$ is the coequalizer of the diagram % https://q.uiver.app/#q=WzAsMyxbMiwwLCJMXzErTF8yIl0sWzQsMF0sWzAsMCwiXFxnYW1tYSJdLFsyLDAsIkZfMSIsMCx7ImN1cnZlIjotMn1dLFsyLDAsIkZfMiIsMix7ImN1cnZlIjoyfV1d
\begin{tikzcd}[ampersand replacement=\&,cramped]
	\gamma \&\& {L_1+L_2} \&\& {}
	\arrow["{F_1}", curve={height=-12pt}, from=1-1, to=1-3]
	\arrow["{F_2}"', curve={height=12pt}, from=1-1, to=1-3]
\end{tikzcd}.

$\gamma$ contains pairs of nodes (hyperedges) coming from $L_1$ and $L_2$, therefore it must belong to $\mathcal{G}$ as defined in the algorithm and $\epsilon : L_1 + L_2 \to X$ must have been constructed by the algorithm. $F : J \to X$ follows from $\epsilon$.

\end{myproof}


Algo.~\ref{algo2} enumerates pre-critical pairs, regardless of them being parallel or not. Taking non-empty subsets in Line~\ref{algo2:line} does not help. The following conjecture would give an optimisation of Algo.~\ref{algo2} so that it yields only non-parallel pre-critical pairs (\ie, critical pairs).
\begin{myproposition}
    Given two rewrite rules $\catspan {L_i} {K_i} {R_i} [f_i] [g_i]$ where $i \in \{1,2\}$,
    a pre-critical pair $L_1 + L_2 \overset{\epsilon}{\twoheadrightarrow} S \xleftarrow{c} J$ is parallel iff the following holds:
    \begin{enumerate}
        \item no edges from $L_1$ and $L_2$ are glued, \ie, have the same image under $\epsilon : L_1 + L_2 \rightarrow S$, and
        \item if two nodes from $L_1$ and $L_2$ are glued, they are in the image of $f_1$ and $f_2$.
    \end{enumerate}
\end{myproposition}
\begin{myproof}[Proof of $\Rightarrow$]
    Suppose we have a parallel pair
\[
% https://tikzcd.yichuanshen.de/#N4Igdg9gJgpgziAXAbVABwnAlgFyxMJZABgBpiBdUkANwEMAbAVxiRAGkB9ARhAF9S6TLnyEU3clVqMWbADI9+gkBmx4CRAMyTq9Zq0QgFAJiVC1oogBYd0-Wy6mB5kRpRluUvbMMBhRc4qwupiyMaknroyBiAAymZBFm7INpF2PiD+TsqqrqHhlFH2hgrcAAQA1GUm-FIwUADm8ESgAGYAThAAtkhkIDgQSBL9dFgMbAAWEBAA1iBFGa0Byh3dSNr9g4jhI2OT03MLMUum1Ax0AEYwDAAKwZaG7VgNEzgJqz2IwwNIAKyBHyQOx+iH+K06nxsmz+1Bwo3Ghims3eEKQADZYVtfrD4ftkQDUYg+iCrAS1ogNiC0WTPsCtlCGFgwDE4BBGVB5iAJjA6BzEGAmAwGDi9oZIMyUeTvltqeDyQB2TEwrk8vn9ADuEG5vIQNKGSsQit2CK5B05jIlhigdDg3I5eu2BqNcNFpuRZyZMWttvqtT4QA
\begin{tikzcd}
K_1 \arrow[r, "f_1", hook] \arrow[d] & L_1 \arrow[rd] \arrow[rrrd] \arrow[r, dashed, hook] & L_1 + L_2 \arrow[d, two heads] & L_2 \arrow[ld] \arrow[llld] \arrow[l, dashed, hook] & K_2 \arrow[l, "f_2"', hook] \arrow[d] \\
C_1 \arrow[rr, hook]                 &                                                     & S                              &                                                     & C_2 \arrow[ll, hook]                 
\end{tikzcd}
\]
\begin{itemize}
    \item Let $v_i \in L_1 + L_2$ for $i = 1,2$ be mapped to the same $v \in S$. Without loss of generality suppose $v_1 \in L_1, v_2 \in L_2$. Since this is a parallel pair, there are mappings $L_2 \rightarrow C_1$, $L_1 \rightarrow C_2$. Let $v_2^{'}$ be image of $v_2$ under such mapping. Because the triangle commutes, $v_2^{'}$ must be sent to $v \in S$. But S is a pushout, so the identified elements of $C_1, L_1$ must be present in K, so $v_1$ has a preimage in K, hence preserved by $f_1$. Similarly for $v_2$.
    \item Suppose there are glued edges. Let $e_i \in L_i$ for $i = 1,2$ be mapped to the same $e \in S$. As before, for the identified vertices we can conclude that they are preserved by $f_1, f_2$ by diagram chasing. By the same argument, it must mean that those edges are preserved by $f_1, f_2$, too, so they must have a pre-image in $K_1, K_2$. But $K_1, K_2$ are discrete, so such edges cannot exist.  
\end{itemize}
\end{myproof}
\begin{myproof}[Proof of $\Leftarrow$]
    Suppose the assumptions 1 and 2 hold, let us prove that the pre-critical pair is parallel. Consider 
    % https://tikzcd.yichuanshen.de/#N4Igdg9gJgpgziAXAbVABwnAlgFyxMJZABgBpiBdUkANwEMAbAVxiRAGkB9ARhAF9S6TLnyEU3clVqMWbADI9+gkBmx4CRAMyTq9Zq0QgFAJiVC1oogBYd0-Wy6mB5kRpRluUvbMMBhRc4qwupiyMaknroyBiAAymZBFm7INpF2PiD+TlIwUADm8ESgAGYAThAAtkhkIDgQSBK1dFgMbAAWEBAA1iBR9obFAcpllUjatfWI4U0t7Z09fRmDptQMdABGMAwACsGWhqVYeW04CSNViI11SACsgedI09eId8PlFzYTt-fvSABs1GerxKv0QNWeVh+o0Q42efz4FD4QA
\[\begin{tikzcd}
K_1 \arrow[r, "f_1", hook] \arrow[d] & L_1 \arrow[rd] &   & L_2 \arrow[ld] & K_2 \arrow[l, "f_2"', hook] \arrow[d] \\
C_1 \arrow[rr]                       &                & S &                & C_2 \arrow[ll]                       
\end{tikzcd}\]
The proof is by diagram chasing.

If no nodes are glued, \ie, $\epsilon: L_1 + L_2 \rightarrow S$ is a monomorphism on nodes, then on nodes $S = L_1 + L_2$. Consider the right square. S is a pushout, so $C_2 \simeq L_1$ on nodes. Then we can define $L_1 \rightarrow C_2$ as identity on nodes and get a parallel pair.

Otherwise, consider two nodes $v_i \in L_i$ for i=1,2 with the same image v in S.
By assumption, $v_2$ is in the image of $f_2 : K_2 \rightarrow L_2$. 

Let $v_2^{'} \in C_2$ be the image of $v_2$ under $K_2 \rightarrow C_2$.

Define the morphism $L_1 \rightarrow C_2$ by sending $v_1$ into $v_2^{'}$ and identity on the rest of the nodes as before and edges as there are no edges from $L_1$ and $L_2$ glued.

By construction that will make the triangle commute.

Similarly for $L_2 \rightarrow C_1$.

The morphisms $C_i \rightarrow S$ are monomorphisms as a pushout of a monomorphism in an adhesive category. 

\end{myproof}

\bibliographystyle{abbrv}
\bibliography{ref}

\end{document}

